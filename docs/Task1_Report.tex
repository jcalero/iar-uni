%%% LaTeX Template
%%% This template is made for project reports
%%%	You may adjust it to your own needs/purposes
%%%
%%% Copyright: http://www.howtotex.com/
%%% Date: March 2011

%%% Preamble
\documentclass[paper=a4, fontsize=12pt]{scrartcl}	% Article class of KOMA-script with 11pt font and a4 format


\usepackage[english]{babel}				% English language/hyphenation
\usepackage[protrusion=true,expansion=true]{microtype}	% Better typography
\usepackage{amsmath,amsfonts,amsthm}			% Math packages
\usepackage[pdftex]{graphicx}				% Enable pdflatex
\usepackage{url}


%%% Custom sectioning (sectsty package)
\usepackage{sectsty}					% Custom sectioning (see below)
\allsectionsfont{\normalfont\scshape}			% Change font of al section commands


%%% Custom headers/footers (fancyhdr package)
\usepackage{fancyhdr}
\pagestyle{fancyplain}
\fancyhead{}						% No page header
%\fancyfoot[L]{\small \url{HowToTeX.com}}		% You may remove/edit this line 
\fancyfoot[C]{}						% Empty
\fancyfoot[R]{\thepage}					% Pagenumbering
\renewcommand{\headrulewidth}{0pt}			% Remove header underlines
\renewcommand{\footrulewidth}{0pt}			% Remove footer underlines
\setlength{\headheight}{13.6pt}


%%% Equation and float numbering
\numberwithin{equation}{section}		% Equationnumbering: section.eq#
\numberwithin{figure}{section}			% Figurenumbering: section.fig#
\numberwithin{table}{section}				% Tablenumbering: section.tab#


%%% Maketitle metadata (Defines how everything above the body should look like: title, header, authors, date, etc..)
\newcommand{\horrule}[1]{\rule{\linewidth}{#1}} 	% Horizontal rule

\title{
\vspace{-1in} 	
\usefont{OT1}{bch}{b}{n}
\normalfont \normalsize \textsc{University of Edinburgh - School of Informatics} \\ [25pt]
\horrule{0.5pt} \\[0.4cm]
\large IAR - Task 1 Report \\
\horrule{1pt} \\[0.5cm]
}
\author{
  \normalfont \normalsize
  Jakob Calero - s0948339\\[-3pt]\normalsize
  Samuel Neugber - s0821562\\[-3pt]\normalsize
  \today
}
\date{}


%%% Begin document
\begin{document}
\maketitle					% Insert the title here
\section{Abstract}
In a known environment and with a simple task one can find all cases and reduce them to simple if statements, doesn't look “smooth” though. Likely to not work that nice for most other cases.

\section{Introduction}
We approached the problem of navigating a small robot in a known environment using two different methods: Discrete, functional behaviour and more continuous, differential control, as described in the first few chapters of Vehicles (FOOT). The Khepera II robots are circular, two-wheeled robots, which come equipped with 8 infra-red sensors arranged as seen in diagram 1. (DIAGRAM) The task was to move a Khepera II robot around in a relatively static and enclosed testing arena (DIAGRAM) while following walls and avoiding obstacles. In order to achieve this behaviour we used discrete conditional cases to determine the current state and act accordingly in one attempt, as well as a more direct mapping between sensor values and motor speeds in another attempt.

\section{Methods}
\subsection{Gathering Data}
For either approach, the control program has to know which values of the IR sensors represent certain thresholds for the distance of the robot to an object. The IR sensors have a range of values between 0 and 1020, where the base level is usually around 50 and putting the robot right next to a white wall gives readings of 1000 and higher. Through experimentation we found that sensor values of 200+ describe a sensible value representing the robot being near an object, regardless of colour and texture of the surface. Additionally, we found that values of 500+ mean that we are too close to an object and values ranging between 180 and 250 are what we are looking for when trying to follow a wall without getting too close or too far away from it (again, regardless of colour/material of the wall). From these values we then extrapolated if necessary.
\subsection{Functional Control}
The first approach we took was to identify cases in which the robot needs to react, capture them as conditionals and then identify appropriate behaviour for these cases. As seen in (DIAGRAM) our final control code for this methodology uses just four if-statements, representing seven conditions:
\begin{itemize}
 \item There is an object to the front-right or too close directly to the right, so turn left.
 \item There is an object to the front-left or too close directly to the left, so turn right.
 \item There is an object to the left, but slightly too far away, so turn left.
 \item There is an object to the right, but slightly too far away, so turn right.
 \item In any other case, move forward.
\end{itemize}
The first two cases use all six IR sensors to avoid obstacles which would get in the way of the robot when moving straight ahead. The next two cases attempt to capture the fact that the robot may turn away from a wall too much when avoiding collision. The robot should therefore rotate back towards the wall if it has moved too far away.
\subsection{Differential Control}
The second approach we took to solve the task ahead was one described in the first few chapters of Vehicles. \cite{1} Our goal was that, ideally, the robot should always be moving and do so as smoothly as possible (if only to look appealing). The control code in (DIAGRAM) shows that in order to do so, each iteration of the control loop resets the wheel speed to a maximum value and then reduces the speed of the wheel on the opposite side of the corresponding IR sensor if certain thresholds are met. The higher the value of the sensor, the more the speed of the corresponding wheel is reduced.
\section{Results}
\subsection{Functional Control}
Using functional control, the behaviour of the robot was close to being deterministic. The robot reliably followed walls if there were no obstacles present. In some cases it stopped a little to late and lightly bumped into the wall or obstacle, but then managed to turn away and continue moving. It managed to avoid obstacles and also stayed close to them if they were long enough had only little curvature. In any other case it just turned away from the object and continued on a straight trajectory. For most cases this behaviour was sufficient to complete the task. In some rare cases on the other hand, the robot would sometimes not behave as intended. One case was that it got stuck between a wall an object, if the object was close enough to the wall and the robot such that the first two rules in the control loop were repeatedly activated in succession. An additional case where the robot would not explore the entire map emerged when objects were placed in a square pattern, resulting in the robot bumping into the first object, turning toward the second, driving there and repeating that process continuously.
\subsection{Differential Control}
Differential control did not work as intended in most cases, so we did not pursue it for long. The robot did turn smoother than it had using functional control, but it was also harder to control. The robot usually turned away too much when avoiding collision, which meant that it rarely ever followed a wall. This method did however have the benefit that the robot did not get stuck in between even the closest gaps, but nicely reduced it's speed and moved through.
\section{Discussion}
In an environment as small and predictable as the one given in this task, good rule based behaviour is possible since most conditions the robot can be found in can be discovered and mapped accordingly. However, the more complex the task and environment get, the harder it will be to achieve such a mapping.
\section{Appendix}

%%% End document
\end{document}
