%%% LaTeX Template
%%% This template is made for project reports
%%%	You may adjust it to your own needs/purposes
%%%
%%% Copyright: http://www.howtotex.com/
%%% Date: March 2011

%%% Preamble
\documentclass[paper=a4, fontsize=12pt]{scrartcl}	% Article class of
KOMA-script with 12pt font and a4 format


\usepackage[english]{babel}				% English
language/hyphenation
\usepackage[protrusion=true,expansion=true]{microtype}	% Better typography
\usepackage{amsmath,amsfonts,amsthm}			% Math packages
\usepackage[pdftex]{graphicx}				% Enable pdflatex
\usepackage{url}
\usepackage[margin=1.5in]{geometry}
\usepackage{algorithm}
\usepackage{algorithmic}
\usepackage{framed}
\usepackage{listings}

\lstset{
language=Matlab,
basicstyle=\footnotesize,
tabsize=2
}


%%% Custom sectioning (sectsty package)
\usepackage{sectsty}					% Custom sectioning (see
below)
\allsectionsfont{\normalfont\scshape}			% Change font of al
section commands


%%% Custom headers/footers (fancyhdr package)
\usepackage{fancyhdr}
\pagestyle{fancyplain}
\fancyhead{}						% No page header
%\fancyfoot[L]{\small \url{HowToTeX.com}}		% You may remove/edit
this line 
\fancyfoot[C]{}						% Empty
\fancyfoot[R]{\thepage}					% Pagenumbering
\renewcommand{\headrulewidth}{0pt}			% Remove header
underlines
\renewcommand{\footrulewidth}{0pt}			% Remove footer
underlines
\setlength{\headheight}{13.6pt}


%%% Equation and float numbering
\numberwithin{equation}{section}		% Equationnumbering: section.eq#
\numberwithin{figure}{section}			% Figurenumbering: section.fig#
\numberwithin{table}{section}				% Tablenumbering:
section.tab#


%%% Maketitle metadata (Defines how everything above the body should look like:
title, header, authors, date, etc..)
\newcommand{\horrule}[1]{\rule{\linewidth}{#1}} 	% Horizontal rule

\title{
\vspace{-1in} 	
\usefont{OT1}{bch}{b}{n}
\normalfont \normalsize \textsc{University of Edinburgh - School of Informatics}
\\ [25pt]
\horrule{0.5pt} \\[0.4cm]
\large IAR - Task 1 Report \\
\horrule{1pt} \\[0.5cm]
}
\author{
  \normalfont \normalsize
  Jakob Calero - s0948339\\[-3pt]\normalsize
  Samuel Neugber - s0821562\\[-3pt]\normalsize
  \today
}
\date{}


%%% Begin document
\begin{document}
\maketitle					% Insert the title here
\section{Abstract}
A dynamically changing task usually requires a dynamic solution. We have therefore successfully revisited the approach of setting motor speeds as a function of the values the sensors give us. The robot will track a lightsource which is sufficiently close while avoiding obstacles and moving in smooth arcs.


\section{Introduction}
In the last task we explored two different control methods for our robot: one which acted on certain threshold-conditions of the infra-red (IR) sensors, and one which directly mapped the values from the sensors to motor speeds. Our findings were that the conditional control approach works well enough in relatively static environments and is a little easier to reason about than the other approach.

This task, on the other hand, required our robot to exhibit more dynamic behaviour in order to follow a light which could frequently change its position. We therefore went back and found a function which uses the sensor values more directly as input to the motor speeds. Using the light sensors we calculate a relative angle the robot should be facing and that angle is then taken as input to the functions which detemine the speed of the individual wheels.

\section{Methods}
\subsection{Gathering Data}
The IR sensors on the Khepera have a range of values between 0 and 1020.
Experimentation showed that when nothing is close to the robot the sensor values
drift at around 50. Sensor values of 200+ describe a sensible value representing
the robot being near an object (within 2cm) while values of 500+ mean that we
are too close to an object. A range between 180 and 250 where the values we
chose as a good distance when trying to follow a wall without getting too close
or too far away from it. From these values we then extrapolated if necessary.
\subsection{Functional Control}
As seen in \emph{figure 3.1} our control code for this methodology uses just
four if-statements.

\begin{figure}[!ht]
\begin{framed}
\begin{algorithmic}
 \IF{FrontSensor is close to object \textbf{OR} RightSensors are close to
object}
 \STATE{Command: Rotate left}
 \ENDIF
 \IF{FrontSensor is close to object \textbf{OR} LeftSensors are close to object}
 \STATE{Command: Rotate right}
 \ENDIF
 \IF{LeftSensor is closer than 250 but further away than 180 from object}
 \STATE{Command: Rotate left}
 \ENDIF
 \IF{RightSensor is closer than 250 but further away than 180 from object}
 \STATE{Command: Rotate right}
 \ENDIF
 \STATE{Command: Move forward}
\end{algorithmic}
\end{framed}
\caption{Functional Control Algorithm}
\end{figure}

Given the order each condition is checked the robot will always turn whenever
there is something directly in front of the it, as illustrated by the first two
conditions above. In the same two conditions the right and left sensors are also
checked but only for a very close distance, triggering only when collision is
imminent and not while travelling along a wall. These two simple cases result in
the behaviour of avoiding object collision and the tendency of following
straight objects (walls) given that as soon as the front sensor is not detecting
anything forward movement is resumed.

The final two conditions attempt to capture the fact that the robot may turn
away from a wall too much when avoiding collision or drift slightly. The robot
should rotate back towards the wall if it has moved too far away. A cap to that
distance is used as we do not want the robot to start rotating when it's
\emph{very} far away from a wall as that might indicate it's not following a
wall to begin with.
\subsection{Differential Control}
The second approach we took to solve the task ahead was one described in the
first few chapters of Vehicles\cite{vehicles}. Our goal was that, ideally, the
robot should always be moving and do so as smoothly as possible to maximise
efficiency and avoid ``stuttering'' movement. The control code in \emph{Appendix
6.2} shows that in order to do so, each iteration of the control loop resets the
wheel speed to a maximum value and then reduces the speed of the wheel on the
opposite side of the corresponding IR sensor if certain thresholds are met. The
higher the value of the sensor, the more the speed of the corresponding wheel is
reduced.
\section{Results}
\subsection{Functional Control}
Using functional control, the behaviour of the robot was close to being
deterministic, reliably following walls with 1-2cm distance. It managed to avoid
obstacles to avoid collision and return to the wall searching pattern
immediately after.

A few problems still remain however. One case was that it got stuck in narrow
passages, such that the first two rules in the control loop were repeatedly
activated in succession. An additional case where the robot would not explore
the entire map if objects were placed in a square-like pattern, resulting in the
robot moving in a loop between the objects without being able to get out.
\subsection{Differential Control}
Differential control did not work as intended in most cases, so we did not
pursue it for long. The robot did turn smoother than it had using functional
control, but it was also harder to control. The robot usually turned away too
much when avoiding collision, which meant that it rarely followed a wall. This
method did however have the benefit that the robot did not get stuck in between
even the closest gaps, but nicely navigated through.
\section{Discussion}
In an environment as small and predictable as the one given in this task, good
rule based behaviour is possible since most conditions the robot can be
extracted and mapped accordingly and good results follow therein, as shown
above. However, the more complex the task and environment get, the harder it
will be to achieve such a design. In these cases, differential control, (or even
better, full proportional–integral–derivative control) and more dynamic basic
behaviour will be useful. Conditional if-cases will run into problems like the
\emph{indecisiveness} of our above problem when navigating narrow corridors and
are generally very messy to maintain.
\section{Appendix}
\subsection{Functional Controller - Source}
\lstinputlisting{controlMain.m}
\subsection{Differential Controller - Source}
\lstinputlisting{controlAlternate.m}
\begin{thebibliography}{9}
\bibitem{vehicles}
  Valentino Braitenberg,
  \emph{Vehicles: Experiments in Synthetic Psychology},
  A Bradford Book,
  1986.
\end{thebibliography}

%%% End document
\end{document}
